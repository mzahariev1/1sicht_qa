\rhead{Umfrage und Ergebnisse}

\section{Umfrage und Ergebnisse}

    In diesem Kapitel soll es um die Gedanken bei der Erstellung, die Durchführung und die Ergebnisse der Umfrage gehen.
    Gerade die Ergebnisse stellen die Meinung zur App von anderen Leuten dar, die zum einen zur Zielgruppe gehören, zum anderen auch noch nie etwas von der App gehört haben.

    Zwischen dem 23.08. und dem 01.09. haben insgesamt 11 Leute teilgenommen.

    \begin{center}
        Die Umfrage kann unter \url{https://forms.gle/riUdkYzGcpsepX4r9} gefunden werden
    \end{center}

    Es war uns wichtig dem Nutzer möglichst viele Möglichkeiten zu geben, sein Feedback in eigenen Worten zu formulieren. Daher gab es zu fast jedem Abschnitt ein Freitext-Feld, die hier immer am Ende des Abschnitts zusammengefasst werden.

    Die Umfrage war in fünf Abschnitte eingeteilt, wobei sich jeder dieser Abschnitte mit einem anderen Bereich der App befasste:

    \newpage

    \subsection{Abschnitt 1: Allgemeines}

        In diesem Abschnitt sollte es um den Eindruck des Nutzers von der App gehen. Außerdem wurde der Nutzer noch nach seiner Meinung zur Idee hinter der App gefragt. Dieses Feedback trägt nichts Konkretes zur Verbesserung der App bei, jedoch erhalten wir generelles Feedback.

        \subsubsection{Allgemeine Bewertung der App}

            Bei dieser Frage ging es um den allgemeinen Eindruck des Nutzers von der App. Hier ist zu sehen, dass der subjektive Eindruck der Nutzer durchschnittlich gut ist.

            \begin{bchart} [min = 0, max = 10, step = 2]
                \bcbar[label=Sehr schlecht]{0}
                \bigskip
                \bcbar{0}
                \bigskip
                \bcbar{3}
                \bigskip
                \bcbar{5}
                \bigskip
                \bcbar[label=Sehr gut]{3}
            \end{bchart}

        \subsubsection{Allgemeine Bewertung der App-Idee}

            Bei dieser Frage ging es um die Idee hinter der App, dass man sich für Klausureinsichten anmelden kann, um Wartezeiten zu verringern.

            Interessant bei dieser Frage ist auch die Diskrepanz zu den Ergebnissen der Frage darüber, die als der Unterschied zwischen der Idee und der tatsächlichen Umsetzung interpretieren kann. Demnach wird die Idee selbst etwas besser eingeschätzt als unsere Umsetzung.

            \begin{bchart} [min = 0, max = 10, step = 2]
                \bcbar[label=Sehr unsinnig]{0}
                \bigskip
                \bcbar{0}
                \bigskip
                \bcbar{0}
                \bigskip
                \bcbar{4}
                \bigskip
                \bcbar[label=Sehr sinnvoll]{7}
            \end{bchart}

        \subsubsection{Bewertung der Benutzbarkeit}

            Diese Frage behandelte erneut ein eher subjektives Gefühl der Nutzer, und zwar wie gut sie sich in der App zurecht fanden und wie gut sie die App benutzen konnten.

            \begin{bchart} [min = 0, max = 10, step = 2]
                \bcbar[label=Nicht benutzbar]{0}
                \bigskip
                \bcbar{0}
                \bigskip
                \bcbar{2}
                \bigskip
                \bcbar{4}
                \bigskip
                \bcbar[label=Sehr benutzbar]{5}
            \end{bchart}

        \newpage

        \subsubsection{Zusammenfassung des allgemeinen Feedbacks zur App}

            In diesem allgemeinen Feedback haben wir, zusammengefasst, zwei große Kritikpunkte erhalten, einmal, dass das Farbschema der App etwas eintönig ist, und weiterhin wurde bemängelt, dass man sich nur mit seinem Google-Account und nicht mit etwa seinem Studentischen Account einloggen kann.

    \subsection{Abschnitt 2: Design und Übersicht}

        In diesem Abschnitt ging es um das Design der App. Zum einen beinhaltet das das Aussehen der App, zum anderen aber auch die Übersichtlichkeit und Platzierungen und Beschriftungen der Buttons. Hier wurde also Feedback zum Aussehen der App erhoben, aber auch wie gut sich Nutzer zurecht finden.

        \subsubsection{Allgemeine Bewertung des Designs der App}

            Bei dieser Frage ging es um die subjektive Bewertung des Designs durch den Nutzer. Auch hier wurde durchschnittlich die Bewertung gut abgegeben.

            \begin{bchart} [min = 0, max = 10, step = 2]
                \bcbar[label=Sehr schlecht]{0}
                \bigskip
                \bcbar{1}
                \bigskip
                \bcbar{3}
                \bigskip
                \bcbar{7}
                \bigskip
                \bcbar[label=Sehr gut]{0}
            \end{bchart}

        \newpage

        \subsubsection{Feedback zur Übersicht der einzelnen Bildschirme}

            Hier sollte es speziell um die einzelnen Bildschirme gehen und wie gut die Nutzer wussten, was sie auf den einzelnen Bildschirmen tun konnten. Diese Frage wurde von fast allen Nutzern mit der besten Bewertung bewertet. Man hat also gesehen, dass sich die meisten Nutzer gut zurecht fanden und alle Funktionen leicht zu sehen waren.

            \begin{bchart} [min = 0, max = 10, step = 2]
                \bcbar[label=Keine Ahnung]{0}
                \bigskip
                \bcbar{0}
                \bigskip
                \bcbar{1}
                \bigskip
                \bcbar{2}
                \bigskip
                \bcbar[label=Totaler Durchblick]{8}
                \end{bchart}

        \subsubsection{Feedback zur Beschriftung der Buttons}

            Bei dieser Frage sollte es um die Beschriftung oder die Symbole auf den Buttons gehen und wie aussagekräftig diese waren. Diese Frage sollte also auch Hinweise darauf geben, wie gut der Nutzer durch die App navigieren konnte. Die Bewertung war hier war quasi deckungsgleich mit der Frage darüber.

            \begin{bchart} [min = 0, max = 10, step = 2]
                \bcbar[label=Keine Ahnung]{0}
                \bigskip
                \bcbar{0}
                \bigskip
                \bcbar{1}
                \bigskip
                \bcbar{2}
                \bigskip
                \bcbar[label=Totaler Durchblick]{8}
            \end{bchart}

        \subsubsection{Feedback zur Menge an Informationen pro Bildschirm}

            Diese Frage beschäftigte sich mit den Informationen pro Bildschirm. Es war uns wichtig, dass die Nutzer nicht überwältigt waren von der Menge an Informationen pro Bildschirm, was auch leicht dazu führen kann, dass sich der Nutzer schwierig zurecht findet. Diese Frage wurde durchschnittlich etwas schlechter beantwortet als die anderen Fragen zum Design.

            \begin{bchart} [min = 0, max = 10, step = 2]
                \bcbar[label=Zu überladen]{0}
                \bigskip
                \bcbar{1}
                \bigskip
                \bcbar{0}
                \bigskip
                \bcbar{4}
                \bigskip
                \bcbar[label=Sehr angemessen]{6}
            \end{bchart}

        \subsubsection{Zusammenfassung des allgemeinen Feedbacks zum Design}

            Generell wurde erneut (Von einem anderen Nutzer als beim allgemeinen Feedback) das Farbschema kritisiert, aber auch, dass manche Buttons etwas zu klein waren für die Menge an Text, die auf ihnen stand.

    \subsection{Abschnitt 3: Registrierung und Anmeldung in der App}

        Da das Einloggen mit einem Google-Account und die Registrierung als Mitarbeiter oder Nutzer essenziell ist für die Benutzung der App, ging es in diesem Abschnitt um die Nutzererfahrung bei diesen Vorgängen.

        \subsubsection{Feedback zur Anmeldung mit dem Google Account}

            Bei dieser Frage ging es um die Anmeldung der Nutzer mit ihrem Google Account. Diese hat bei allen Nutzern funktioniert, wenn auch bei einem mit mehreren Versuchen.

            \begin{bchart} [min = 0, max = 10, step = 2]
                \bcbar[label=Ja]{10}
                \bigskip
                \bcbar[label={Ja, mit mehreren Versuchen}]{1}
                \bigskip
                \bcbar[label=Nein]{0}
                \bigskip
                \bcbar[label=Sonstiges]{0}
            \end{bchart}

        \newpage

        \subsubsection{Feedback zur Registrierung in der App}

            Die Registrierung in der App funktionierte auch bei allen Nutzern, bis auf einem (Welcher nicht der war, der auch die Probleme mit dem Google Login hatte).

            \begin{bchart} [min = 0, max = 10, step = 2]
                \bcbar[label=Ja]{10}
                \bigskip
                \bcbar[label={Ja, mit mehreren Versuchen}]{1}
                \bigskip
                \bcbar[label=Nein]{0}
                \bigskip
                \bcbar[label=Sonstiges]{0}
            \end{bchart}

    \subsection{Abschnitt 4: Anmeldung zu einer Klausureinsicht}

        In diesem Abschnitt ging es nun um die zentrale Idee unserer App: Die Anmeldung zu einer Klausureinsicht. Der Nutzer sollte hier bewerten wie klar es ihm war für welche Klausureinsicht und Zeitslots er sich anmelden kann.

        \subsubsection{Feedback zur Übersicht bei der Auswahl einer Klausureinsicht}

            Bei dieser Frage sollte es um die Anmeldung für eine Klausureinsichten gehen und wie einfach sich diese gestaltet. Insbesondere wie eindeutig es für den Nutzer ist, für welche Klausureinsicht er sich gerade anmeldet.

            \begin{bchart} [min = 0, max = 10, step = 2]
                \bcbar[label=Absolut unklar]{0}
                \bigskip
                \bcbar{0}
                \bigskip
                \bcbar{2}
                \bigskip
                \bcbar{4}
                \bigskip
                \bcbar[label=Sehr klar]{5}
            \end{bchart}

        \subsubsection{Feedback zur Übersicht bei der Auswahl eines Zeitslots}

            Nun ging es zusätzlich um die Anmeldung zu einem bestimmten Zeitslot.

            \begin{bchart} [min = 0, max = 10, step = 2]
                \bcbar[label=Absolut unklar]{0}
                \bigskip
                \bcbar{0}
                \bigskip
                \bcbar{1}
                \bigskip
                \bcbar{5}
                \bigskip
                \bcbar[label=Sehr klar]{4}
            \end{bchart}

        \subsubsection{Zusammenfassung des Feedbacks zu Klausureinsicht-Anmeldung}

            Zur Anmeldung der Klausureinsichten gab es zwei Kritikpunkte: Einmal wurde die Länge der Timestamps bei der Auswahl eines Zeitslots kritisiert und ein weiterer Nutzer fand, dass man bei der Auswahl der Klausureinsicht auch schon im Bildschirm davor den Ort der Klausureinsichten sehen sollen könnte.

    \newpage

    \subsection{Abschnitt 5: Bugs und Probleme}

        Im letzten Abschnitt ging es nun um die Probleme und Bugs, die Nutzer während ihrer Zeit mit der App bemerkt haben. Hier lag der Fokus auf Freitext-Antworten, damit der Nutzer frei beschreiben kann, wie reibungslos die Benutzung der App aussah, oder auch den Umfang der Probleme beschreiben kann.

        Da hier relativ wenige Antworten gegeben wurden, lässt sich vermuten, dass die Nutzer entweder auf keine Probleme oder Bugs gestoßen sind, oder dass sie so gering waren, dass der Nutzer nicht die Zeit investieren wollte um sie am Ende festzuhalten.

        \subsubsection{Zusammenfassung des Feedbacks zu Bugs}

            Ein Nutzer gab uns das Feedback, dass die Geschwindigkeit der App für ihn zu gering war. Da es sich dabei um eine sehr subjektive Aussage handelt und er nicht konkretisiert hat, ob er sich auf die Anmeldung, den Wechsel zwischen Activities oder sonstigem bezieht, konnten wir hieraus leider keine Verbesserungen ableiten.

        \subsubsection{Zusammenfassung des Feedbacks zu generellen Problemen}

            In diesem Abschnitt wurden keine Freitext-Antworten abgegeben.
